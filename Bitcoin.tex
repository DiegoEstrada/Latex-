\documentclass[12pt,letterpaper]{article}
\usepackage[utf8]{inputenc}
\usepackage[spanish]{babel}
\usepackage{amsmath}
\usepackage{amsfonts}
\usepackage{amssymb}
\usepackage{graphicx}
\usepackage[left=2cm,right=2cm,top=2cm,bottom=2cm]{geometry}
\author{Diego Estrada Granados}
\title{Instituto Politécnico Nacional}
\begin{document}

\maketitle
\begin{center}
\textbf{Bitcoin}
\end{center}

\section*{Introducción}
Inicialmente se utilizaba el trueque como forma de intercambio de productos, pero el problema de este radicaba fundamentalmente en que las personas que lo realizaran debían tener deseo y necesidad de cambio lo cual a veces era difícil, por esto se dio paso a el pago con oro o plata, pero este era muy pesado lo cual generaba inseguridad por este motivo se busco algo más liviano y cómodo, dando origen a los llamados “billetes” el primero en utilizarlos fue el emperador Mongol KUBALIKHAN en el siglo XI.Actualmente existen diferentes medios con los cuales comerciar, tarjetas de crédito y monedas electrónicas. 

\section*{¿Qué es Bitcoin?}
Imaginemos un libro de cuentas en internet en el que se anota en todo momento quien es el propietario de una serie de fichas virtuales denominadas bitcoins. 
Los dueños de esas fichas pueden transferirlas a otras personas, de tal forma que en el libro se registra la transferencia y quién es el nuevo dueño.Esta concepción nos permite comprender como funciona el Bitcoin.

Bitcoin es una moneda electrónica descentralizada, concebida en 2009 por quien se ha dado a conocer como Satoshi Nakamoto (aunque su verdadera identidad se desconoce). A diferencia  de la mayoría de las monedas, el funcionamiento de Bitcoin no depende de una institución central, sino de una base de datos distribuida. El software ideado por Nakamoto emplea la criptografía para proveer funciones de seguridad básicas, tales como la garantía de que los bitcoins sólo puedan ser gastados por su dueño, y nunca más de una vez.

\section*{Seguridad}
Quizás el mayor logro de Satoshi Nakamoto sea el de haber resuelto el problema del doble gasto en un sistema descentralizado, que tanto ha desvelado a economistas y programadores. Para evitar que un mismo bitcoin sea gastado más de una vez por la misma persona (en otras palabras, para evitar la falsificación), la red se vale de lo que Nakamoto describe como un servidor de tiempo distribuido, que identifica y ordena secuencialmente las transacciones e impide su modificación. Esto se logra por medio de pruebas de trabajo encadenadas (las cuales se muestran como “confirmaciones”). 

Cabe destacar que, hasta la fecha, no se ha documentado ningún caso de doble gasto, pero es cierto que un ataque informático de este tipo es teóricamente posible, siempre y cuando el atacante controle al menos el 51 por ciento del poder computacional que protege a la red. Según los expertos, gracias a la arquitectura criptográfica de Bitcoin, una transferencia entre direcciones Bitcoin es al menos tres veces más segura que una transferencia entre cuentas bancarias (sin contar el riesgo que implica la forzosa intromisión de terceros en el sistema bancario).

La naturaleza P2P de la red Bitcoin hace imposible el establecimiento de un control centralizado de todo el sistema.
Esto impide el aumento arbitrario de la cantidad de bitcoins en circulación (lo que generaría inflación) y cualquier otro tipo de manipulación del valor por parte de las autoridades.

\section*{¿Puede Bitcoin ser un buen dinero?}
Hemos visto cómo funciona Bitcoin, qué implicaciones tiene y cuáles son sus principales novedades. Pero hay una pregunta que sigue siendo pertinente y que hay que analizar. ¿Puede realmente un sistema informático como Bitcoin ser un buen dinero?.


Para que algo funcione como dinero primero tiene que arrancar como medio de intercambio. Uno de los debates que está generando en ciertos círculos económicos es si Bitcoin cumple el denominado teorema de regresión del dinero, enunciado por el economista austriaco Ludwig von Mises.
El teorema, de forma resumida, dice que para que un bien empiece a usarse como medio de intercambio es necesario que tenga una demanda no monetaria previa para fijar el precio inicial desde donde arrancar.
Uno de los problemas de Bitcoin es que, a primera vista, parece que no cumple con el teorema. 

Hasta la fecha Bitcoin no se ha comportado como un dinero propiamente dicho. Es cierto que se usa como medio de pago para algunas transacciones en internet, pero no es usado de forma generalizada. 
La mayor parte de su demanda no es monetaria, sino especulativa. La cotización de Bitcoin en las casas de cambio se ha comportado hasta la fecha de manera muy  volátil. Es decir, lo contrario a lo que se espera de un buen dinero, que debería ser el bien de valor más estable de toda la economía.  

El hecho de que Bitcoin llegue a ser el dinero del futuro no dependerá de cómo está construido estructuralmente, de su diseño, ya que cumple con los requisitos.
Dependerá de si los particulares y empresas están dispuestos a usarlos, y de si los gobiernos logran frenarlo. En todo caso será un experimento extraordinario.

\section*{Referencias}
\begin{verbatim}
-http://www.revista-anales.es/web/n_19/pdf/seccion_6.pdf
-https://www.certsi.es/sites/default/files/contenidos/estudios/doc/int_bitcoin.pdf
-https://bibliolibertaria.org/files/bitcoin-la-moneda-del-futuro.pdf
\end{verbatim}	

\end{document}