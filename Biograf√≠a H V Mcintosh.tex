\documentclass[10pt,letterpaper]{article}
\usepackage[utf8]{inputenc}
\usepackage[spanish]{babel}
\usepackage{amsmath}
\usepackage{amsfonts}
\usepackage{amssymb}
\usepackage{graphicx}
\usepackage[left=2cm,right=2cm,top=2cm,bottom=2cm]{geometry}
\author{Diego Estrada Granados}
\title{Instituto Politécnico Nacional }
\begin{document}
\maketitle
\begin{center}
Biografía de Harold V. McIntosh 
Grupo: 2CM4\linebreak
\end{center}
\section{Introducción}
Harold Varner. McIntosh Martin fue un extraordinario matemático que
desarrolló su vida académica en nuestro país. Especializado en computación,
fue uno de los principales impulsores de esta disciplina en México, a tal grado
que varias de nuestras instituciones más importantes actualmente
dedicadas a la computación son fruto de su trabajo académico.
Formó parte del Departamento de Física del Centro de Investigación y de
Estudios Avanzados (Cinvestav) del IPN de 1964 a 1965; habiendo dirigido las
tesis de licenciatura de Adolfo Guzmán Arenas y de Raymundo Segovia
Navarro. Cabe señalar que la tesis de Guzmán Arenas, titulado CONVERT, fue
el primer artículo mexicano de computación publicado en una revista
internacional.[1]
\section{Trayectoria}
Harold V. McIntosh obtuvo la Licenciatura en Ciencias con especialidad en Física de Colorado A\&M College en 1949, la Maestría en Ciencias (en Matemáticas) de la Universidad de Cornell en 1952, y terminó créditos doctorales en Cornell y Brandeis; obtuvo el Doctorado de Filosofía en Química Cuántica en la Universidad de Uppsala en 1972.[2]

MacIntosh trabajó desde 1964 hasta 1965 en el Departamento de Física de la CIEA del IPN, ahora Cinvestav (Centro de Investigación y Estudios Avanzados del Instituto Politécnico Nacional); el diseño e implementación del lenguaje de programación CONVERT tuvieron lugar durante este período. De 1965 a 1966, MacIntosh fue director del departamento de programación en el Centro de Computación Electrónica de la Universidad Nacional Autónoma de México, donde diseñó y desarrolló el lenguaje de programación REC.

Durante los siguientes nueve años, MacIntosh era un profesor de la Escuela de Física y Matemáticas (ESFM) del IPN, donde se convirtió en coordinador del grupo de Matemáticas Aplicadas. Bajo su dirección, los compiladores de REC se construyeron para los equipos más nuevos que llegan al CENAC del IPN (Centro Nacional de Computación), y él personalmente paquetes de software desarrollados para su uso en los diferentes cursos que enseñaba. De catorce tesis de licenciatura que dirigió en la ESFM, se destaca por haber sido publicado como tres artículos separados en la revista Journal of Mathematical Physics , uno de los cuales  describe un problema en una clase que ahora se llama "sistemas micz Kepler", las iniciales de pie para McIntosh, Cisneros y Zwanziger. También de este periodo, el papel de MacIntosh simetría y la degeneración  se citó con entusiasmo en tres ocasiones en la segunda edición del reconocido libro de la mecánica clásica de Herbert Goldstein.  Tomó una licencia de un año a partir de ESFM en 1972 para obtener su Ph.D. en química cuántica, que fue galardonado con la máxima distinción, en la Universidad de Uppsala. Las contribuciones de MacIntosh a la resonancia Grupo Uppsala llevaron a varias tesis de doctorado en Suecia.

Entre 1970 y 1975 MacIntosh fue también un consultor en el Instituto Nacional de Energía Nuclear (INEN, ahora ININ), y en 1975 él y su grupo se trasladó a la Universidad Autónoma de Puebla (UAP), donde fundó el Departamento de aplicaciones de la microcomputadora en el Instituto de Ciencias de la UAP, y donde permaneció hasta su muerte. Dio cursos a estudiantes de informática de la Facultad de Ciencias Físicas y Matemáticas de la UAP y en los últimos dos decenios y medio su interés se volvió al estudio de los autómatas celulares, en la que también se convirtió en un experto reconocido.[3] 

\section{Referencias }
[1] Escuela Superior de Física y Matemáticas (2015), Harold V. McIntosh [Online] Disponible:
http://www.esfm.ipn.mx/Documents/Harold-V.pdf

[2] Pedro Hernandez, diciembre 2013, La influencia deH. V. McIntosh [Online] Disponible :
http://delta.cs.cinvestav.mx/~mcintosh/comun/gto91/node2.html

[3] Gerardo Cisneros y Erikki Brandas, 30 de noviembre de 2015, Harold V. McIntosh [Online]
Disponible: http://scitation.aip.org/content/aip/magazine/physicstoday/news/10.1063/PT.5.6193

\end{document}