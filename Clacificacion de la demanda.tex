\documentclass[11pt,letterpaper]{article}
\usepackage[utf8]{inputenc}
\usepackage[spanish]{babel}
\usepackage{amsmath}
\usepackage{amsfonts}
\usepackage{amssymb}
\usepackage{graphicx}
\usepackage[left=2cm,right=2cm,top=2cm,bottom=2cm]{geometry}
\author{Diego Estrada Granados\\Grupo: 2CM4}
\title{Instituto Politécnico Nacional }
\begin{document}
\maketitle
\section{Demanda}
La demanda se define como la cantidad y calidad de bienes y servicios que pueden ser adquiridos en los diferentes precios del mercado por un consumidor (demanda individual) o por el conjunto de consumidores (demanda total o de mercado), en un momento determinado. 

\section{Distntas formas de la demanda}
\subsection{En relaión a su satisfacción}
\subsubsection*{Demanda insatisfecha}
Se trata de una demanda con gran repercusión pero poca disponibilidad del mismo, por ejemplo hospitales o salud publica.
\subsubsection*{Demanda satisfecha}
Se encuentram  productos o servicios de gran disponibilidad con diversas características y muchos tipos diferentes de ofertantes, es el caso de celulares y objetos tecnológicos.
\subsubsection*{Demanda saturada}
Estas demandas se encuentran establecidas en los mercados de manera permanente por su gran cantidad. Es el caso de las legumbres básicas como el frijol  o habas.
\subsubsection*{Demanda no saturada}
Estas demandas son consumidas de manera masiva de acuerdo al precio y promociones que contengan los productos, es el caso de la comida rápida.
\subsection{En relación a su necesidad}
\subsubsection*{Demanda de bienes social y nacionalmente necesarios}
Estos bienes son los más importantes e irreemplazables, la sociedad debe consumirla para satisfacer sus necesidades básicas e indispensables y el estado debe encargarse de proveerla, es el caso del agua.
\subsubsection*{Demanda de bienes no necesarios}
Los individuos los consumen para satisfacer deseos o caprichos, por determinadas modas o promociones. En estos bienes pueden encontrarse servicios de belleza, por ejemplo.
\subsection{En relación con su temporalidad}
\subsubsection*{Demanda continua}
Son todas las demandas necesarias duerante los cuatro ciclos del año, puede ser el caso de la ropa.
\subsubsection*{De ciclo o estacionalidad}
Durante determinado periodo del año, generalmente se trata de servicios que los individuos consumen, como vacaciones en determinado lugar, paquetes turísticos, etc.
\subsection{En relación a su destino}
\subsubsection*{De bienes finales}
El producto a consumir ya esta listo, no es necesario la preparación de ningún tipo de objeto extra para su consumo, puede aparecer diferentes frutas en almíbar.
\subsubsection*{De bienes industriales}
Estos bienes necesitan cierta preparación anterior para su consumo, necesitando previamente algún tipo de objeto extra. Los pasteles pueden ser un ejemplo de esto, requieren leche, huevo, azucar y hornos para elaborarse.
\section{Curva de la demanda}
La demanda puede ser expresada gráficamente por medio una curva. La pendiente de la curva determina cómo aumenta o disminuye la demanda ante una disminución o un aumento del precio. 

En relación con la elasticidad, la demanda se divide en tres tipos:

\subsection*{Elástica}
Cuando la elasticidad de la demanda es mayor que 1, la variación de la cantidad demandada es porcentualmente superior a la del precio.
\subsection*{Inelástica}
Cuando la elasticidad de la demanda es menor que 1, la variación de la cantidad demandada es porcentualmente inferior a la del precio.
\subsection*{Elasticidad unitaria}
Cuando la elasticidad de la demanda es 1, la variación de la cantidad demandada es porcentualmente igual a la del precio.

\section{Referencias electrónicas}
\subsubsection*{http://www.elmundo.com.ve/diccionario/demanda--ley-de-la-demanda.aspx}
\subsubsection*{http://www.tipos.co/tipos-de-demanda/}
\end{document}