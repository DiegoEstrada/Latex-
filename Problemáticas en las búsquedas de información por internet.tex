\documentclass[12pt,letterpaper]{article}
\usepackage[utf8]{inputenc}
\usepackage[spanish]{babel}
\usepackage{amsmath}
\usepackage{amsfonts}
\usepackage{amssymb}
\usepackage{graphicx}
\usepackage[left=2cm,right=2cm,top=2cm,bottom=2cm]{geometry}
\author{Diego Estrada Granados\\Grupo: 2CM4}
\title{Instituto Politécnico Nacional}
\begin{document}
\maketitle
\section*{Problemáticas en las búsquedas de información por internet con fines académicos}


La educación en los últimos veinte años ha sufrido cambios radicales en especial por el aparecimiento y el uso del internet, del cual nuestras actividades sociales, recreativas, escolares e inclusive institucionales se ven inmersas por esta herramienta. Sin embargo, el acceso a grandes cantidades de información en un periodo de tiempo muy reducido no se han significado un impacto positivo en nuestras maneras de obtener información, pues ahora resulta más sencillo obtener fuentes de información de cualquier procedencia y otorgadas por cualquier persona que desee expresar algún mensaje en la red, razón por la cual eventualmente no verificamos la veracidad de esa información y a veces ni siquiera sabemos quién o quiénes son los responsables de difundir dicha información.


La creciente cantidad de información almacenada en la red de redes dificulta la búsqueda efectiva que deseamos realizar, es por ello que como estudiantes de nivel superior debemos de ser capaces de identificar los sitios en los cuales la información que se proporciona cuanta con referencias confiables para ser utilizada y sobre todo, es la requerida para el trabajo que estamos desarrollando, afortunadamente nuestros profesores identifican de manera oportuna nuestras carencias académicas y nos proponen alternativas a esta problemáticas sugiriéndonos técnicas para ingresar nuestra búsqueda en algún motor de búsqueda, además se nos brindan puntos importantes para evaluar la calidad que un  sitio web, remarcando la importancia de respetar el trabajo de sus autores, por ultimo ofrecen direcciones electrónicas las cuales consideran son relevantes y confiables en el ámbito escolar.


\section*{Conclusión}

Como estudiantes es de suma importancia manejar información de alta confiabilidad respaldada por personas u organismos con un antecedente en el ámbito correspondiente, por lo tanto, nos corresponde a nosotros modificar las técnicas de búsqueda que empleamos ya que, en un futuro, podrían marcar diferencia con otros colegas al realizar nuestro trabajo y además podría ser un área de oportunidad para las futuras generaciones.

\section*{Obtenido de:}
http://eprints.uwe.ac.uk/14120/1/indexJournals-final.pdf
\end{document}