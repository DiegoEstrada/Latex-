\documentclass[12pt,letterpaper]{article}
\usepackage[utf8]{inputenc}
\usepackage[spanish]{babel}
\usepackage{amsmath}
\usepackage{amsfonts}
\usepackage{amssymb}
\usepackage{graphicx}
\usepackage[left=2cm,right=2cm,top=2cm,bottom=2cm]{geometry}
\author{Diego Estrada Granados}
\title{Instituto Politécnico Nacional }


\begin{document}
\maketitle
\begin{center}
\textbf{Internet de las Cosas}
\end{center}

\section*{Introducción}
Después de la red de redes (World Wide Web, WWW) y del Internet móvil, estamos
inmersos en una tercera, y potencialmente más disruptiva, fase: el llamado Internet de las
Cosas (Internet of Things, IoT).
IoT hace referencia a un mundo conectado hasta el último extremo, donde objetos y seres
físicos interaccionan con entornos virtuales de datos en el mismo espacio y tiempo.
Soñamos con poder medir y controlar por completo nuestro entorno. Esto será posible
usando la información extraída a través de millones de sensores que poblarán cada rincón
de nuestro entorno y que podrán estar integrados en cualquier objeto de nuestra vida
cotidiana. Sin embargo, conseguir esto va a requerir una estructura tremendamente
compleja: Internet de las Cosas. Su planificación y ejecución están dando ya sus primeros
pasos.

\section*{IdC}
Al igual que con varios conceptos novedosos, las raíces de IdC se pueden remontar al
Instituto de Tecnología de Massachusetts (MIT).
Según el Grupo de soluciones empresariales basadas en Internet (IBSG, Internet
Business Solutions Group) de Cisco, IdC es sencillamente el punto en el tiempo en el que se
conectaron a Internet más “cosas u objetos” que personas. El número de cosas conectadas a internet sobrepasó en 2008 el número de
habitantes del planeta. Se estima que habrá 50.000 millones de dispositivos
conectados en 2020.
\\
\\

\begin{figure}[hbtp]
\centering
\includegraphics[width=7cm]{1.PNG}
\caption{Internet de las cosas nació entre los años 2008 y 2009}
\end{figure}

Actualmente, IdC está compuesta por una colección dispersa de redes diferentes y con
distintos fines. n. A medida que IdC evoluciona, estas redes y muchas otras estarán
conectadas con la incorporación de capacidades de seguridad, análisis y administración. Esta inclusión permitirá que IdC sea una herramienta aún más poderosa. 

\begin{figure}[hbtp]
\centering
\includegraphics[width=7cm]{2.PNG}
\caption{IdC se puede considerar la red de redes}
\end{figure}

\section*{Aplicaciones del IdC}
Al atravesar el umbral de conectar a Internet más objetos que personas, se abrió una enorme
ventana de oportunidades para la creación de aplicaciones en las áreas de la automatización,
el uso de sensores y la comunicación entre máquinas. De hecho, las posibilidades son casi
infinitas. Además de que ahora convergen facotores como: Popularización de placas de HW libre ,abaratamiento de sensores, Mejora comunicacionesetc.

\begin{figure}[hbtp]
\centering
\includegraphics[width=10cm]{3.PNG}
\caption{Algunos ejemplos del IdC}
\end{figure}

\section*{Desafios y bareras del IdC}
No obstante, son varias las barreras que podrían retrasar el desarrollo de IdC. Las tres
barreras de mayor magnitud son la implementación de IPv6, la energía para alimentar los
sensores y el acuerdo sobre las normas.
\subsection*{Implementación del IPv6} 
En febrero de 2010, se agotaron las direcciones IPv4 del mundo , esta situación podría lentificar el
progreso de IdC, ya que los posibles miles de millones de sensores necesitarán direcciones
IP exclusivas. Además, IPv6 facilita la administración de las redes gracias a las capacidades
de autoconfiguración y ofrece características de seguridad mejoradas.

\subsubsection*{Energía para los sensores}
Para que IdC alcance su máximo potencial, los sensores deberán
ser autosustentables.  Lo que se necesita es una forma de que los sensores generen electricidad a partir de elementos
medioambientales como las vibraciones, la luz y las corrientes de aire.

\subsection*{Normas}
Si bien se han realizado grandes progresos en cuanto a las normas, se necesita aún
más, especialmente en las áreas de seguridad, privacidad, arquitectura y comunicaciones.
IEEE es solo una de las organizaciones que actualmente trabajan para sortear estas
dificultades, con la tarea de garantizar que los paquetes de IPv6 se puedan direccionar a
través de tipos de red diferentes.
Es importante destacar que, si bien existen barreras y desafíos, no son insalvables. En vista
de los beneficios de IdC, estos problemas serán resueltos. Solo es cuestión de tiempo. 

\section*{Referencias}

\begin{itemize}
\item $http://www.cisco.com/c/dam/global/es_mx/solutions/executive/assets/pdf/internet-of-things-iot-ibsg.pdf$
\item $https://www.carriots.com/newFrontend/img-carriots/press_room/Construyendo_un_proyecto_de_IOT.pdf$
\item $https://actualidad.madridnetwork.org/imgArticulos/Documentos/635294387380363206.pdf$
\item $http://www.iingen.unam.mx/es-mx/Publicaciones/GacetaElectronica/Mayo2015/Paginas/Internetdelascosas.aspx$
\end{itemize}





\end{document}